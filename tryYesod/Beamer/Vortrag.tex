\documentclass{beamer}
\usepackage{beamerthemeshadow}
\usepackage[utf8]{inputenc}
\begin{document}
\title{Software Transactional Memory}  
\author{Robert Kupferschmied, Matrikel 59112}
\date{\today} 
\frame{\titlepage} 

\frame{\frametitle{Inhaltsverzeichnis}\tableofcontents[sections={1-2}]} 
\frame{\frametitle{Inhaltsverzeichnis}\tableofcontents[sections={3}]}

\section{Einführung}
\subsection{Prozess, Thread, Spark} 
\frame{\frametitle{Prozess, Thread, Spark}
\begin{itemize}
\item Prozess: Algorithmisch ablaufende Informationsverarbeitung\\
	Prozesse können unterbrochen werden\\
	\begin{description}
		\item[Bereit:] Besitzt alle Ressourcen, wartet auf Prozessorzuordnung
		\item[Laufend:] Ist aktuell dem Prozessor zugeordnet
		\item[Wartend:] Ist durch Betriebssystem unterbrochen wartet auf Betriebsmittel
	\end{description}
	\item Thread: Leichtgewichtiger Prozess\\
		Teilt sich Betriebsmittel mit anderen Threads,\\ besitzt eigenen Stack\\
	\item Spark: leichtgewichtiger Thread, wird vom Laufzeitsystem gestartet\\
\end{itemize}	
}
\subsection{Verschränkung}
\frame{
	\frametitle{Verschränkung}
	Threads können gleichzeitg auf gleiche Betriebsmittel zugreifen.\\
	\centering
	\vspace{0.5cm}
	\begin{tabular}{|l|l|}
		\hline Java Code & Pseudo Assembler \\ 
		\hline i++;  & mov a,0x00 //Adresse der Variable \\
 					 & inc a\\
 					 & mov 0x00,a\\
		\hline 
	\end{tabular}
}

\frame{\frametitle{Anzahl möglicher Verschränkung} 
	\begin{figure}
\centering
\includegraphics[width=3.5cm,height=3.5cm]{../Vortrag/Verschraenkung}
\caption{Mögliche Wege}
\label{fig:MgWege}
\end{figure}
\begin{itemize}
	\item 6! = 24 mögliche Wege\\
	\item Testen ist sehr aufwendig!!!
\end{itemize}
}
\subsection{Threadsynchronisation}
\frame{\frametitle{Threadsynchronisation}
	\textbf{Blockierung}\\
	\begin{itemize}
	\item Implementation durch Semaphore\\
	\item von Dijkstra konzipiert\\
	\end{itemize}
	Semaphor l -$>$ 2 Operationen: \begin{enumerate}
	\item l.lock
	\item l.unlock
	\end{enumerate}
	\qquad \\
	In Java: \\
	\qquad Object lock;\\
	\qquad Synchronized(lock)\{ ... \};\\
}
\frame{\frametitle{Mutex Variablen in Haskell}
	Mutex = mutual exclusion = Wechselseitiger Ausschluss\\
	\qquad \\
	\begin{itemize}
	\item	takeMVar :: MVar a -$>$ IO a   -- Block when MVar is empty\\
	\item	putMVar :: MVar a -$>$ a -$>$ IO () --Block when MVar is full\\
	\item	tryTakeMVar :: MVar a -$>$ IO (Maybe a) –Noneblocking\\
	\item	tryPutMVar :: MVar a -$>$ a -$>$ IO Bool  --Noneblocking\\
	\end{itemize}
	
}
\frame{\frametitle{Deadlock / Dining Philosophers}
	\begin{figure}
	\centering
	\includegraphics[width=0.7\linewidth,height=6.5cm]{../Vortrag/578px-Dining_philosophers}
	\caption{Dining Philosophers}
	\label{fig:578px-Dining_philosophers}
	\end{figure}
 
}
\frame{\frametitle{Atomare Operationen/Feingranular}
	\begin{itemize}
	\item atomar = unteilbar\\
	\end{itemize}
	\textbf{Atomic-Bibliothek in Java} \\
	Bsp: \\
	\qquad AtomicLong al = new AtomicLong(0);\\
	\qquad al.getAndIncrement(); //ist ein atomarer Maschinenbefehl\\
}
\section{Software Transactional Memory}
\frame{\frametitle{Software Transactional Memory}
	\begin{itemize}
	\item	Technik zur Vereinfachung von nebenläufiger Programme\\
	\item	Erlaubt Operationen die den Status von Variablen ändern\\
	\item	Diese Änderungen werden atomar ausgeführt\\
	\item	Man kann mehrere Operationen zu atomaren Blöcken zusammenfassen\\
	\item	Die Transaktionen werden durch Transaktionsmanager durchgeführt\\
	\end{itemize}
}
\frame{\frametitle{Ablauf einer Transaktion} 
	\begin{itemize}
	\item 	Transaktionen sind atomare Blöcke\\
	\item 	Es gibt einen Transaktionsmanager\\
	\item 	Transaktionsmanager besitzt einen Log\\
	\item 	Daten werden in Log geschrieben\\
	\item 	Im Log werden die Veränderungen gespeichert\\
	\item 	Nach Transaktion werden die Änderungen im Log im Hauptspeicher atomar übernommen ("commit", vgl. SQL)\\
	\item 	Es wird nur ein "commit" ausgeführt wenn Daten mit Log konsistent mit HS, ansonsten muss Transaktion wiederholt werden
	\end{itemize}
	
}
\subsection{Die Bibliothek - Control.Concurrent.STM}
	\frame{\frametitle{Die Bibliothek - Control.Concurrent.STM}
	\begin{itemize}
		\item atomically:: STM a -$>$ IO a\\
		\item newTVar:: a -$>$ STM (TVar a)\\
		\item readTVar:: TVar a -$>$ STM a\\
		\item writeTVar:: TVar a -$>$ a -$>$ STM ()\\
		\item retry:: STM a\\
		\item orElse:: STM a -$>$ STM a -$>$ STM a\\
		\item throwSTM:: Exception e =$>$ e -$>$ STM a\\
		\item catchSTM:: Exception e =$>$ STM a -$>$ (e -$>$ STM a) -$>$ STM a\\
	\end{itemize}
}
\subsection{Eine Transaktion}
	\frame{\frametitle{Eine Transaktion}
		atomInc :: Num a =$>$ TVar a -$>$ STM ()\\
		atomInc x = do \\
		\qquad i $<$- readTVar x -- lesen des aktuellen Status\\
		\qquad writeTVar x (i+1) -- Veränderung im Log\\
		main = do \\
		\qquad shared $<$- atomically \$ newTVar 0\\
		\qquad atomically \$ atomInc shared\\
		\begin{itemize}
		\item atomically ruft den Transaktionsmanager auf
		\item Keine IO Aktion in Transaktion möglich 
		\end{itemize}
		
	}
\subsection{Blocking – Retry}
\frame{\frametitle{Blocking – Retry}
	retry :: STM a\\
	\begin{itemize}
	\item retry blockiert den aktuellen Spark bis sich die die Variablen ändern\\
	\end{itemize}
	Bsp:\\ 
	takeForks f1 f2 = atomically \$ do\\
	\qquad b1 $<$- readTVar f1\\
	\qquad b2 $<$- readTVar f2\\
	\qquad if(b1 \&\& b2) then( do\\
	\qquad writeTVar f1 False\\
	\qquad writeTVar f2 False) else retry
}
\subsection{orElse}
\frame{\frametitle{orElse}
	orElse    :: STM a -$>$ STM a -$>$ STM a\\
	\qquad \\
	Wirkung: \\
	\begin{enumerate}
	\item Die erste Transaktion wird durchgeführt. Falls sie ein Ergebnis zurückliefert, endet orElse\\
	\item Falls die erste Transaktion retry aufruft. Wird die zweite Transaktion durchgeführt\\
	\end{enumerate}
}
\subsection{Yesod und STM}
\frame{\frametitle{Yesod und STM}
\begin{itemize}
	\item STM ist eine eigenständige Bibliothek\\
	\item Webserver sind parallele Programme\\
	\item Jeder Client bekommt sein eigenen Thread\\
	\item Vorteilhaft gemeinsamen Speicher über STM zu synchronisieren:\\
	\begin{enumerate}
	\item Keine/kaum Deadlocks
	\item Keine Verschränkungseffekte
	\end{enumerate}
\end{itemize}
	
}
\section{Vor- und Nachteile}
\frame{\frametitle{Vor- und Nachteile}
	STM bietet eine Reihe von Vorteilen.\\
	\begin{enumerate}
	\item zusammensetzbare Atomare Blöcke
	\item nicht Blockierend
	\end{enumerate}
	Frage: Warum brauche ich noch MVars?\\
	\begin{itemize}
	\item takeMVar ist schneller als takeTMVar\\
	\item Code wird aber nicht automatisch schneller durch die Benutzung von MVars!!!\\
	\end{itemize}
}
\subsection{Fairness}
\frame{\frametitle{Fairness}
\begin{itemize}
	\item MVars sind fair\\
	\item Mehrere Threads greifen auf MVar zu, dann werden die Threads im Fifo Prinzip abgearbeitet\\
	\item STM hat diese Möglichkeit nicht\\
	\item Hier werden alle wartenden Threads "aufgeweckt", wenn sich die Variable ändert\\
	\item Keine Echtzeitanwendungen!!!
\end{itemize}
	
}
\subsection{Performance}
\frame{\frametitle{Kosten von Transaktionen}
	\begin{description}
		\item[writeTVar] übernimmt nur Änderung im Log\\
		\item[readTVar] muss das Protokoll durchgehen -$>$ Prüfung ob es von früherem writeTVar beschrieben wurde\\readTVar hat Kosten von O(n), n = Anz. der Log Einträge\\
		\item[retry] bracht den Log, hier werden readTVars ausgewertet
		\item[commit] 1.(Nicht Konsistent) Verwerfen einer Transaktion ist billig\\ 2.(Konsistent) Sperren aller beteiligten TVars und schreiben des Logs
	\end{description}
}
\frame{\frametitle{Faustregeln zur Verbesserung}
\begin{enumerate}
\item Immer eine feste Anz. von TVars lesen. Sonst enstehen Kosten von O(n$^{2}$).\\
\item Transaktionen sollten gleich lang sein.\\
\end{enumerate}
Bsp:\\
atomically \$ mapM takeTMVar ts //Kosten O(n$^2$)\\
mapM (atomically . takeTMVar) ts //Kosten n*O(n)
}
\subsection{Hardware Transactional Memory}
\frame{\frametitle{Hardware Transactional Memory}
\textbf{Problem:} Prozessor-Cache behindert Parallelität.\\
\begin{itemize}
\item Transaktionsmanager muss Log in HS auswerten.\\
\end{itemize}
\textbf{Idee:}
\begin{itemize}
	\item jeder Prozessor erhält 2 Caches, einen für Transaktionen und einen regulären, diese sind exklusiv
	\item transaktionale Cache erhält zusätzliche Logik, die commit und retry erleichtern
	\item vorläufiges Schreiben nicht im HS sichtbar.
\end{itemize}
}
\subsection{Quellen}
\frame{\frametitle{Quellen}
	\begin{itemize}
		\item http://chimera.labs.oreilly.com/books/1230000000929/\\ch10.html\#sec\_stm-cost\\
		\item https://github.com/simonmar/parconc-examples\\
		\item https://github.com/yesodweb/yesod/wiki/Keeping\\-(in-memory)-state-with-yesod\\
		\item http://de.wikipedia.org/wiki/Transactional\_Memory\\
		\item http://rosettacode.org/wiki/Dining\_philosophers
		\item http://www.cs.brown.edu/~mph/HerlihyM93/\\herlihy93transactional.pdf
	\end{itemize}
}

\subsection{Zusammenfassung}
\frame{\frametitle{Zusammenfassung}
Pro:
\begin{itemize}
	\item Umfangreiche Bibliothek zur Threadsynchronisierung
	\item einfache Benutzung, ohne Deadlocks oder Verschränkungseffekte
	\item Möglichkeit atomare Blöcke zu definieren
\end{itemize} 
Contra: 
\begin{itemize}
	\item STM ist nicht fair
	\item Lange Transaktionen committen nie.
\end{itemize} 
}
\frame{\frametitle{weitere Beispiele}
\begin{itemize}
	\item STM Channel
	\item Raucherproblem (TMVars)
	\item STM-Scala
\end{itemize}}
\end{document}